\section{Discussion}

% 1) how separating scaffolding, pbx and application makes development easier and more sustainable

\subsection{Supporting Design Exploration and Sustainability through Modularity}

A key feature of \ONT{} is its separation of a platform's component parts into five distinct layers: \textit{UI}, \textit{Scaffolding}, \textit{Application}, \textit{System} and \textit{Data}. By separating a synchronous telephony platform's implementation out into these layers, \ONT{} allows for the partial decoupling of the systems' key components from each other. A key benefit of this separation of concerns is reusability: different combinations of interactions can be implemented in the scaffolding layer, often without the need to update the rest of the system.

This separation is particularly beneficial to the System Layer: while PBXServices such as FreeSWITCH are freely available, installing and configuring them can be complicated and time consuming. 
Having such a limited number of SystemActions which can be performed by the PBXService provides the System Layer with a core feature-set: one which is unchanging between applications, but can be combined with logic from the Scaffolding and Application Layers to support a wide variety of interaction archetypes. For example, the research workshop and speed dating scenarios each re-used the same SystemActions (`\textit{createRoom}' and `\textit{transfer}') to produce markedly different experiences, without the need to change the PBXService. 

In contrast, existing synchronous group telephony platforms (as described in their publications) seem to have been explicitly designed to implement a single engagement format \cite{Kazakos2016, Talhouk2017, Yadav2019}. As such, the implementation of these platforms is tightly coupled to the `talk show' format, and implementing new interaction methods would require significant re-engineering.
While introducing new AppActions in \ONT{} will likely always require the development of additional features within the ApplicationService (e.g. the production of a queue system, or an implementation of round-robin), this can still be achieved without touching the PBXService: meaning that developers of such features don't necessarily need to deploy or have a technical understanding of complicated platforms like FreeSWITCH.
As a result, we argue that the reusability supported by an approach like \ONT{} results in a much lower barrier to the design and prototyping of platforms for synchronous group telephony.

% 2) enabling new types of synchronous IVR engagements through scaffolding

\subsection{Scaffolding New Interactions through Synchronous Group Telephony}

While the talk show format supported by previous synchronous telephony projects can be effective at delivering information to groups of participants, it offers limited opportunities for participant interaction and agency as, by necessity, participants spend most of the call muted. Not only is this incongruous with most participants' prior experiences of phone calls (which typically consist of unstructured discussion between two individuals), but it is also incompatible with any use-cases which require open discussion between participants, or for all participants (including the host) to be on equal footing. Despite these limitations, the use of alternatives to the talk show format within synchronous group telephony has not been adequately explored. As discussed, we believe that this is partially because prior platforms seem to have been designed around talk shows, and would require significant re-engineering to test new formats or be adapted in response to changing stakeholder requirements. However, we also posit that the lack of well defined and commonly agreed axioms can make designing, communicating, and visualising the processes of these engagement formats unwieldy.

We argue that the example scenarios presented in this paper using the visual vocabulary demonstrate that \ONT{} is flexible enough to support the exploration, development and communication of a wide variety of group interaction archetypes through synchronous telephony. The use of \ONT{} to scaffold a talk show format (Figure \ref{fig:talkshow}), complete with an automated pre-show routine (Figure \ref{fig:preshow}), demonstrates that it can be used to describe the engagements facilitated by existing synchronous group telephony platforms, and further highlights the value of a vocabulary which is able to describe and communicate these complex systems. Furthermore, during our experiences developing these scenarios through the design vocabulary there were multiple points where we had to consider unanticipated system requirements, or gaps in logic. For example, when defining the timed Gate to reconvene breakout groups (Figure \ref{fig:breakout}), we realised that there should be a manual override available to the host in case they wanted to reconvene the groups early. Such instances highlight that this format has value in being used for developing low-fidelity logical prototypes.

This paper has also demonstrated how the Scaffolding and App Layers can be used to produce new engagement models, beyond the capabilities of previous synchronous group telephony platforms. For example, while prior platforms limited participants to the largely passive role of an audience member, the research workshop example (Figure \ref{fig:breakout}) supports all participants engaging in open discussion through the use of breakout rooms, without requiring anyone to be muted. The speed dating example (Figure \ref{fig:roundrobin}) demonstrates how this could be taken further: by implementing the `round-robin' algorithm  in the ApplicationService to determine participant pairings, it demonstrated the potential for programmatic scaffolding to produce remote encounters where all participants have comparable levels of agency. This could be expanded even further to support highly structured engagement formats, such as Robert's Rules of Order: a parliamentary procedure used in many contexts to support group discussion and decision making with `\textit{due regard for every member's opinion}' \cite{robert2020}. In such a case, the ApplicationService acts as the arbiter of the engagement---able to direct proceedings, potentially with minimal intervention from the Host. 

Such possibilities suggest that synchronous telephony offers potential for facilitating group interactions beyond those covered in the scenarios presented in this paper. We argue that this potential can be more easily explored through \ONT{}: that the visual design vocabulary supports the design, communication, and analysis of a wide array of new engagement formats hitherto unutilised in remote contexts with offline participants.

% 3) using the layers to explore each as their own design space

\subsection{Accessing Under-Explored Design Spaces}

One of the key benefits of a common vocabulary is that it can be used to share an understanding of a domain which can be communicated across people, computers or both \cite{struder1998}. As \ONT{}'s design vocabulary is more approachable, less technical, and more easily communicable than the underlying architectures and systems it describes, it offers designers and engineers a bridge for discussion and exploration of the potential interactions offered by synchronous group telephony. Furthermore, as this visual format is grounded by a formal ontology, it opens a design space that can be referenced, shared, and built upon by different systems---independent from their underlying software stack. Such an approach lowers the barrier to entry to the exploration, experimentation and appropriation of the uses of synchronous group telephony in new engagement contexts and use cases.

The layered framing of \ONT{} also supports approaching each of these layers as their own design spaces, independent of the rest of the platform's infrastructure. The Data Layer provides a strong example: while explorations have been made regarding how systems can support participatory recordkeeping practices \cite{rolan2017}, the HCI community has performed little to no research into how organisations performing voice-based stakeholder engagements can (or should) approach recordkeeping practices. Researchers within this design space could explore how informed consent should be taken, recorded or rescinded through voice-based platforms; or how resulting audio records could be created, stored, accessed, documented, appraised, and disposed of through voice-based interfaces. Examples of what explorations could be made within the Application Layer include investigating how programmatic, agent-based systems (such as chatbots \cite{Jain2018, Yadav2019} or virtual participants \cite{Bartindale2021}) could be incorporated into the synchronous activities offered by these platforms. As such, we posit that the layers of \ONT{} are rich design spaces, within which we can explore how other practices and technologies can interact with the medium of synchronous group telephony.