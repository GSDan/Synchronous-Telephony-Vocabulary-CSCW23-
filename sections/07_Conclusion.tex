\section{Future Work}

We are currently in the process of incorporating \ONT{} into a demonstrable synchronous telephony platform. Within our future work, we plan to use \ONT{} to support the kinds of group interaction formats illustrated in this paper's scenarios, explore some of the previously discussed recordkeeping concepts within the Data Layer, and investigate how \ONT{} could also be used to scaffold asynchronous interactions before, during and after synchronous engagements \cite{Yadav2017}. Through the implementation of this platform, we hope to be able to provide other researchers in this space recommendations for how to put \ONT{} into practice, as well as further refine the vocabulary in response to findings resulting from its implementation.

It is also worth noting that much of \ONT{} could be applied to other mediums, such as synchronous video, through digital standards such as WebRTC. We wished to focus on its application to group telephony within this paper, given that it is a particularly under-explored design space. However, as \ONT{} is a formal ontology, future work could extend it to suit other mediums.

\section{Conclusion}

Recent years have demonstrated the value and potential of synchronous telephony platforms which enable engagements with remote, offline groups of participants through traditional phone networks. However, this potential has been under-explored, with all published platforms in this space following the same `talk show' engagement format. In response, this paper presents \ONT{}: a design vocabulary, grounded in a formal ontology describing the components of a platform needed to support the dynamic scaffolding of group interactions through synchronous telephony. Demonstrating through a series of scenarios describing three different use cases of synchronous group telephony, we argue that \ONT{} is not only capable of describing the interactions enabled by previously published systems, but also new interaction formats with a flexibility previously unavailable within remote telephony. We argue that \ONT{} provides an approachable, structured vocabulary which offers a more sustainable framework for system design, and highlights that the layers of components within these systems each offer a rich, under-explored design space.